\documentclass[12pt]{amsart}

% \usepackage{sectsty}
\usepackage{graphicx}

% % Margins
% \topmargin=-0.45in
% \evensidemargin=0in
% \oddsidemargin=0in
% \textwidth=6.5in
% \textheight=9.0in
% \headsep=0.25in

\title{Load Balancing: The Secret to High-Performance Implementations on GPU}
\author{Muhammad Osama}
\date{November 1\textsuperscript{st}, 2022}

\begin{document}
\maketitle	

Fine-grained workload and resource balancing is the key to high performance for regular and irregular computations on the GPUs. This seminar will present my dissertation work titled \emph{``Load Balancing on the GPU''}, a GPU load-balancing programming model for fine-grained computations.

I will first present the proposed abstraction for irregular workloads that decouples load balancing from work processing and aims to support both static and dynamic schedules with a programmable interface to implement new load-balancing schedules. Prior to my work, the only way to unleash the GPU's potential on irregular problems has been to workload-balance through application-specific, tightly coupled load-balancing techniques. With the open-source framework for load-balancing, we hope to improve programmers' productivity when developing irregular-parallel algorithms on the GPU, and also improve the overall performance characteristics for such applications by allowing a quick path to experimentation with a variety of existing load-balancing techniques.

Using the insights from load-balancing irregular workloads, I will then show how we build Stream-K, a work-centric parallelization of matrix multiplication (GEMM), and related computations in dense linear algebra. Whereas contemporary decompositions are primarily tile-based, our method operates by partitioning an even share of the aggregate inner loop iterations among physical processing elements. This provides a near-perfect utilization of computing resources (resource balancing), regardless of how efficiently the output tiling for any given problem quantizes across the underlying processing elements. Finally, I will discuss future research directions on GPU load balancing.

\end{document}